\documentclass[]{elementary-physics}

\title{5. Increasing Energy}
\date{2023-01-05 v0.1}

\begin{document}

\maketitle

\tableofcontents

\section{Abstract}

[TODO rename file]

\section{Background}

In physics and chemistry, the law of conservation of energy\cite{energy} states that the total energy of an isolated system remains constant; it is said to be conserved over time.

This implies that the sum of all powers in the system equals to zero which is guaranteed by Kirchhoff's circuit laws\cite{Kirchhoff} and Newton's 3rd law of motion\cite{Newton}.

But what about energy transfer? Is the amount of energy leaving one system always the same as the amount of energy entering another? Would double-entry\cite{DoubleEntry} bookkeeping on power always balance?

For instance; consider the equation of a harmonic oscillator\cite{oscillator}, a LRC-circuit driven by an AC-source expressed as a sum of voltages

\begin{equation}
Lq'' + Rq' + C^{-1}q - sin(\omega t) = 0
\end{equation}

Once solved we can change the equation from a sum of voltages to a sum of powers by multiplying with $q'$

\begin{equation}
Lq''q' + Rq'q' + C^{-1}qq' - sin(\omega t)q' = 0
\end{equation}

But two equations are missing; one that contains $-Rq'q'$ where the heat is going to and one that contains $sin(wt)q'$ where the power running the generator is coming from, now we have a system of equations:


\begin{subequations}
\begin{align}
other.terms - Rq'q' = 0 \\
Lq''q' + Rq'q' + C^{-1}qq' - sin(\omega t)q' = 0 \\
other.terms + sin(\omega t)q' = 0
\end{align}
\end{subequations}


We will ignore the heat and focus on the electro-mechanicals; a wheel with magnets and stationary coils. The two equations involved are Faraday’s Law\cite{Faraday} $V = Blv$ on the electrical side and Lorentz force\cite{Lorentz} $F = BlI$ on the mechanical side. They eace replace one of the two $sin(\omega t)$ where B is now a function $B(\theta)$.

We get

\begin{subequations}
\begin{align}
Lq'' + Rq' + C^{-1}q - Bl\theta' = 0 \\
J\theta'' + r\theta' + Blq' = 0
\end{align}
\end{subequations}

where $J$ is the inertia of the wheel and $r$ is the friction. If we change the equations to powers

\begin{subequations}
\begin{align}
Lq''q' + Rq'q' + C^{-1}qq' - Bl\theta'q' = 0 \\
J\theta''\theta' + r\theta'\theta' + Blq'\theta' = 0
\end{align}
\end{subequations}

we have a system where the two equations communicate energy in a balanced way. In this paper we question that it is always so, experiments suggests that the two $B$-functions are usually but not always the same.

\section{Simulations}

To aid us in our research we have used a free simulator called FEMM\cite{FEMM} which can be scripted using Lua 4.0\cite{Lua}. There is also a command-line version of FEMM called xfemm\cite{xfemm} that we have used heavily.

The femm programs allows us to create snapshots of set-ups with coils and magnets. The simulation then provides us with torque and flux-linkage.

[TODO check if still valid and rewrite for latex]

 1: select a wire-diameter D
 2: calculate 
    number of turns N = 56\% * coil-area / wire-area
    resistance R = pCu * 1000 * wire-length / wire-area 
      where pCu = 1.72E-8
    currrent for 1 W, I = sqrt(1 / R)
 3: test the coil twice (with and without current) for each angle
 4: get impulse T = the sum of all torque (with current minus without current) times delta-angle (integrate)
 5: get the flux-linkage when no current
 6: get the flux-linkage divided by delta-angle to get the voltage (derivate)
 7: get impulse V = the sum of all voltage times delta-angle (integrate)
 8: assuming linearity we calculate for each coil
    Tk = T/N*I
    Vk = V/N
    Rk = R/N\^2
    Nk = 56\% * coil-area * 4/pi
    VpT = Vk / Tk
 9: VpT > 1 is a generator-coil and VpT < 1 is a motor-coil
10: we can now choose any wire diameter D and calculate
    N = Nk/D\^2
    T = Tk*N*I
    V = Vk*N
    R = Rk*N\^2
11: for each pair of coils, mo and gen, with any wire, moD and genD, calculate a relative torque for comparisom between coil-pairs
    T = moT - genT = (moTk*moN - genTk*genN)*I
      where I = (genV - moV) / (genR + moR)



\section{Algorithms}

[TODO Check code and algorithms in Lua and FEMM]
[Maybe by Odd Larsson?]

\section{Experiments}

[TODO Validate the simulations with experiments]

\section{Conclusion}

[TODO]

\appendix

\section{In Plain English}

[TODO]

\section{På Ren Svenska}

[TODO]

\input{last-comment.tex}

[TODO]

\printbibliography

\end{document}
